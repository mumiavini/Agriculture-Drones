\documentclass[a4paper,12pt]{article}

\usepackage[utf8]{inputenc}
\usepackage[T1]{fontenc}
\usepackage[brazil]{babel}
\usepackage{amsmath}
\usepackage{amssymb}
\usepackage{siunitx}
\usepackage{geometry}
\geometry{a4paper, left=3cm, right=2cm, top=3cm, bottom=2cm}
\usepackage{setspace}
\usepackage{indentfirst}
\usepackage{graphicx}
\usepackage{float} 
\usepackage{booktabs}
\usepackage{array}
\usepackage{tabularx}
\usepackage{hyperref}
\hypersetup{
    colorlinks=true,
    linkcolor=black,
    filecolor=magenta,      
    urlcolor=blue,
    citecolor=black,
    pdfencoding=auto
}

\title{\textbf{PIBIT}\\ °\\Analise Comparativa de Motores e Hélices de uso em Aeronaves agrícolas de Asa Fixa}
\author{Vinicius Andrade Trento}
\date{\today}
\onehalfspacing

\begin{document}

\maketitle
\newpage
\tableofcontents
\newpage


\section{Contextualização e Definição dos Requisitos de Engenharia}

O presente relatório constitui a execução da fase de pesquisa fundamental delineada no Plano de Trabalho PIBIT, com foco específico no desenvolvimento de soluções para o projeto NAPI - Aeronaves de Pequeno Porte.

A missão central consiste em subsidiar a engenharia de sistemas para o AGRO-VANT, uma plataforma de asa fixa destinada a operações agrícolas que exigem robustez para cargas úteis elevadas (categorias \textit{Heavy} e \textit{Super Heavy}) e autonomia estendida.

A transição de aeronaves leves para plataformas pesadas (MTOW > \SI{25}{kg}) e superpesadas (MTOW > \SI{150}{kg}) impõe uma mudança paradigmática na seleção de componentes. Diferente dos drones de pequeno porte, onde a simplicidade e o baixo custo dominam, a classe pesada exige certificação de confiabilidade, redundância crítica e eficiência termodinâmica otimizada.

\subsection{O Envelope Operacional Agrícola}
O ambiente agrícola apresenta desafios singulares que definem os requisitos de propulsão:

\begin{itemize}
    \item \textbf{Ciclos de Carga:} Ao contrário de voos de cruzeiro estáveis, a aplicação agrícola envolve ``tiros'' de pulverização, curvas de reversão apertadas e variações constantes de altitude (seguimento de terreno), exigindo resposta rápida do acelerador (\textit{throttle response}).
    \item \textbf{Atmosfera Hostil:} A presença de material particulado, umidade elevada e vapores químicos corrosivos exige que motores e aviônica possuam blindagem (IP ratings elevados) e filtragem de ar superior.
    \item \textbf{Densidade de Altitude:} Operações no interior do Brasil frequentemente ocorrem em dias quentes e altitudes consideráveis, reduzindo a densidade do ar ($\rho$). Isso afeta diretamente a mistura ar-combustível em motores aspirados e a eficiência de empuxo das hélices, exigindo margens de potência (\textit{power reserve}) superiores a 30\%.
\end{itemize}

\section{Análise Aprofundada de Motores de Combustão Interna (ICE)}

Para aeronaves de asa fixa de grande porte destinadas a longas durações de voo (endurance > 2 horas), a densidade energética dos hidrocarbonetos permanece insuperável pela tecnologia de baterias atual. A gasolina aeronáutica (Avgas) ou automotiva premium oferece cerca de \SI{12000}{Wh/kg}, enquanto as melhores baterias de Lítio atingem \SI{250}{Wh/kg} a \SI{300}{Wh/kg}.

\subsection{Motores de Ciclo Otto de 2 Tempos (2T)}
Os motores de dois tempos representam o padrão industrial para drones na faixa de \SI{25}{kg} a \SI{120}{kg} devido à sua simplicidade mecânica e alta relação potência-peso (\textit{Power-to-Weight Ratio}).

\subsubsection{Arquitetura Mecânica e Dinâmica}
A ausência de válvulas de admissão e escape (substituídas por janelas no cilindro) e a ignição a cada revolução do virabrequim conferem aos motores 2T uma capacidade de entrega de torque imediata.

\begin{itemize}
    \item \textbf{Configuração Boxer (Flat-Twin):} Para a aplicação no AGRO-VANT, a configuração de cilindros opostos (Boxer) é a recomendação técnica primária. O movimento simultâneo dos pistões para fora e para dentro cancela as forças de inércia de primeira ordem, reduzindo drasticamente a vibração transferida para a fuselagem.
    \item \textbf{Sistemas de Admissão:} Motores de alta performance utilizam válvulas de palheta (\textit{reed valves}) no cárter. Elas otimizam o fluxo de mistura em baixas rotações e impedem o refluxo, melhorando o torque em regimes parciais – essencial para manter a velocidade de pulverização constante.
\end{itemize}

\subsubsection{Gerenciamento Térmico e Lubrificação}

\subsection{Motores de Ciclo Otto de 4 Tempos (4T)}
Para a categoria ``Super Heavy'' (>\SI{150}{kg}) ou quando a eficiência de combustível é prioritária sobre a potência bruta, os motores 4T ganham relevância.

\subsubsection{Eficiência Volumétrica e Consumo}
Os motores 4T possuem um ciclo de admissão e escape dedicado, permitindo uma queima muito mais completa do combustível.

\begin{itemize}
    \item \textbf{Fuel Consumption} Um motor 2T típico consome \SI{450}{g/kWh} a \SI{600}{g/kWh}. Um motor 4T moderno com injeção pode atingir \SI{280}{g/kWh} a \SI{350}{g/kWh}.
    \item \textbf{Confiabilidade:} A lubrificação por cárter úmido ou seco garante uma película de óleo constante, resultando em TBO (\textit{Time Between Overhaul}) de 1000 a 2000 horas.
\end{itemize}

\subsection{Injeção Eletrônica (EFI) vs. Carburação}
A pesquisa aponta uma tendência irreversível para sistemas EFI em drones pesados.

\begin{itemize}
    \item \textbf{Carburadores:} Incapazes de compensar variações de densidade do ar em tempo real.
    \item \textbf{EFI:} A ECU ajusta o tempo de injeção em microssegundos, mais eficaz e garante economia maior de combustível.
\end{itemize}

\section{Pesquisa de Sistemas de Propulsão Elétrica de Alta Performance}

Embora limitados em autonomia total, os sistemas elétricos oferecem confiabilidade mecânica superior e controle de torque instantâneo.

\subsection{Benefícios dos Motores Elétricos x Combustão}
\begin{itemize}
    \item \textbf{Eficiência Energética:} Motores elétricos atingem eficiência de 90\% a 95\%, enquanto motores de combustão típicos operam entre 25\% e 35\%.
    \item \textbf{Controle Instantâneo de Torque:} O motor elétrico entrega torque máximo imediatamente, ideal para aceleração rápida e manobras.
    \item \textbf{Manutenção Reduzida:} Sem válvulas, sem carburador, sem sistemas complexos de lubrificação.
    \item \textbf{Silêncio Operacional:} Menor ruído em operação, reduzindo impacto sonoro no ambiente agrícola.
\end{itemize}
\subsection{Tecnologias de Motor Elétrico}
\begin{itemize}
    \item \textbf{Motores Brushless DC (BLDC):} Preferidos pela alta relação potência-peso e eficiência. Requerem controladores eletrônicos (ESC) avançados.
    \item \textbf{Controle por FOC (Field-Oriented Control):} Permite controle preciso de torque e velocidade, otimizando a eficiência em toda a faixa operacional.
\end{itemize}

\section{Hélices: Aerodinâmica e Materiais para Carga Pesada}

\subsection{Seleção de Material e Construção}
A escolha do material impacta a assinatura de vibração do sistema propulsivo.

\begin{table}[H]
\centering
\caption{Comparativo de Materiais de Hélices para Drones Pesados}
\label{tab:helices}
\small
\begin{tabularx}{\textwidth}{@{}lXXXXX@{}}
\toprule
\textbf{Material} & \textbf{Rigidez} & \textbf{Peso} & \textbf{Vibração} & \textbf{Resistência} & \textbf{Aplicação} \\ \midrule
Madeira Laminada & Média & Médio & Alta Absorção & Baixa (Quebra segura) & Motores a combustão (amaciamento) \\
Fibra de Carbono (CFRP) & Altíssima & Baixo & Baixa (Transmite) & Alta (Pode danificar eixo) & Elétricos e 2T balanceados \\
Nylon/Carbono & Baixa & Médio & Média & Média & Drones menores (inadequado >25kg) \\ \bottomrule
\end{tabularx}
\end{table}

\section{Sensoriamento e Telemetria Avançada}

Para garantir a segurança operacional e o monitoramento detalhado da aeronave, a seguinte suíte de sensores deve ser integrada:

\subsection{Monitoramento de Fluidos e Temperatura}
\begin{itemize}
    \item \textbf{Sensor de Fluxo de Combustível (Flowmeter):}
    Em vez de medir o nível do tanque (que oscila durante o voo), este sensor mede o consumo real instantâneo. Isso permite que o sistema calcule a autonomia restante com alta precisão.

    \item \textbf{Termopares (Sensores de Temperatura):}
    Devem ser instalados dois tipos de sensores térmicos para proteção do motor:
    \begin{itemize}
        \item \textit{Sensor de Cabeçote (CHT):} Monitora se o motor está superaquecendo fisicamente, prevenindo travamento do pistão.
        \item \textit{Sensor de Exaustão (EGT):} Monitora a temperatura dos gases de escape. Uma mudança brusca aqui alerta sobre falhas na queima (mistura pobre ou falha na vela) antes que o motor pare.
    \end{itemize}
\end{itemize}

\subsection{Monitoramento de Vibração}
\begin{itemize}
    \item \textbf{Acelerômetro:}
    Utiliza-se o acelerômetro para ``escutar'' a saúde mecânica.
    \item \textbf{Diagnóstico Automático:} O sistema identifica padrões de vibração que indicam se uma hélice está desbalanceada ou se os rolamentos internos do motor estão desgastados, alertando para manutenção preventiva.
\end{itemize}

\newpage

\begin{thebibliography}{99}
   

    % Materiais de Hélices (Nylon vs Carbono)
    \bibitem{tiruvenkadam2024}
    Tiruvenkadam, N., et al. (2024). \textit{Investigation of Structural and Thermal Analysis of Drone Propeller Materials}. Journal of Physics: Conference Series, 2925, 012002. IOP Publishing. Disponível em: doi:10.1088/1742-6596/2925/1/012002[cite: 1962, 1963, 1982, 1984]

    % Motores de Combustão Interna (Fundamentação teórica e dados de consumo)
    \bibitem{heywood1988}
    Heywood, J. B. (1988). \textit{Internal Combustion Engine Fundamentals}. New York: McGraw-Hill Education.

\end{thebibliography}

\end{document}
